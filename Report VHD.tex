\documentclass{article}
\usepackage[english,activeacute]{babel}
\usepackage[latin1]{inputenc}
\usepackage{amsmath,amsfonts,amssymb,amstext,amsthm,amscd}
\usepackage{graphicx}
\usepackage{here} %usar el comandao[H] Para fijar la posición de las imágenes
\usepackage{multicol}
\graphicspath{{Imagenes/}}
% Estilo de página y encabezados 
% --------------------------------------------------------------------------------------------------------------------------------------------------------------
\usepackage{fancyhdr}
\usepackage{xcolor} %Definir colores personalizados
\usepackage{listings} %Sirve para pegar codigo fuente de programas
\usepackage{caption} %Agregar titulos a los codigos
\usepackage{karnaugh-map}
\usepackage{hyperref}
\hypersetup{
    colorlinks=true,
    linkcolor=blue,
    filecolor=magenta,      
    urlcolor=blue,
    pdftitle={Overleaf Example},
    pdfpagemode=FullScreen,
    }

\urlstyle{same}
\usepackage{ragged2e}

%%
%%
\setlength{\headheight}{15.2pt}
\usepackage[paperwidth=8.5in, paperheight=11.0in, top=1.0in, bottom=1.0in, left=1.0in, right=1.0in]{geometry}  
\pagestyle{fancyplain}
\fancyhead[R]{LRT2022 Digital design}
\fancyhead[C]{}
\fancyhead[L]{Fall 2022}
\fancyfoot[L]{}
\fancyfoot[C]{\thepage}
\fancyfoot[R]{}
% --------------------------------------------------------------------------------------------------------------------------------------------------------------
% Inicio del documento
\begin{document}
\fancypagestyle{plain}{
   	\renewcommand{\headrulewidth}{1pt}
   	\renewcommand{\footrulewidth}{1pt}
}
\renewcommand{\footrulewidth}{1pt}
\renewcommand{\tablename}{Tabla}
% --------------------------------------------------------------------------------------------------------------------------------------------------------------
%Titulo y autor
\author{}% No llenar, el documento debe ser anónimo..
\title{Individual assignment}
\date{LIS}%Escriba su carrera
\maketitle
% --------------------------------------------------------------------------------------------------------------------------------------------------------------
% Escribir resumen 150-200 palabras
\begin{abstract}
In this paper will be reported the application of problems using the Basys 3 board where programmed files will be implemented by means of the Vivado software, these programs will be related to the combinational logic circuit designs that are presented in the corresponding section. These circuits were developed through EDA Playground where the present activity was carried out, which consisted of putting a real-life practical problem into practice with the tools that are at hand. This activity meant the summary of the first period where we used all the knowledge about digital design to date, in this report we will see the analysis of the procedure, the use of truth tables, karnaugh maps, Boolean functions and logic gates, program design and application on the board, in order to solve the practical problems presented.


\end{abstract}

% --------------------------------------------------------------------------------------------------------------------------------------------------------------
\begin{multicols}{2} %Documento a dos columnas
\section*{Introduction}\label{seccion}  
Two problems will be presented hereafter, these will show practical cases of real life whose solution must be carried out through the use of combinational circuits. First, it is requested to design a circuit capable of controlling the filling of water in a cistern  through a hydraulic system composed of a tank, a motor and a cistern. In second place will be requested to design a circuit capable of detecting what kind of coins are inserted into a vending machine that accepts only three types of coins.
\begin{itemize}
    \item \textbf{Cistern}: In this problem the objective is to find a way to design a system that is able to control the filling of a cistern. There are four sensors, A and B which represent high and low water levels respectively and C and D which represent high and low cistern levels.
    \item \textbf{Coin detector}: In this problem we seek to find a way to design a system that is capable of detecting the types of currency inserted in a vending machine. This machine only accepts coins of the following denominations: \$1, \$5 and \$10 pesos.
\end{itemize}

\section*{Design}
It will be described how the design of the solution was carried out through the explanation of the inputs, outputs. Analysis of the the procedure to get to the code, use of maps and programming style.
\subsection*{Cistern}

To develop the solution for the tank, 4 inputs were used, which are represented by the states in which the sensors are found, either detecting (1) or not detecting (0). Input A represents high level in the water tank. Input B represents low level in the water tank. Input C represents high level in the cistern. Input D represents low level in the cistern.
\\ \phantom\\ \\ \phantom \\ 
Now to represent the outputs, in the problem it is requested that the motor is turned on when the tank is not empty and the tank is not full, this case must be represented with M = 1, on the other hand it must be kept off when the tank is full, in this case M = 0.
\\ \phantom\\ 
Additionally, two more output were added representing error in the sensors "E" and "L", since there are situations that should not occur, such as having the maximum filling sensor on and the minimum filling sensor off, since both should be on for the reading to be correct. E stands for error, and L stands for filling the cistern.
\\ \phantom\\ \\ \phantom \\ 
The procedure used to arrive at the code was to implement the inputs and outputs mentioned above, with this information a design was executed in VHDL and with the help of Karnaugh maps the most simplified form of each of the Boolean functions was obtained and assigned to its corresponding output. In this way it was noted when an output was 1 or 0 together with its corresponding meaning. 
\subsubsection*{Karnaugh maps}

\underline{KMAP for output M:} \vspace{-4mm}

\begin{karnaugh-map}[4][4][1][][][][]
        \minterms{1,3,5,7}
        \terms{}{$x$}
        \autoterms[0]
        \implicant{1}{7}
        \draw[color=black, ultra thin] (0, 4) --
    node [pos=0.7, above right, anchor=south west] {$CD$} % Y label
    node [pos=0.7, below left, anchor=north east] {$AB$} % X label
    ++(135:1);
\end{karnaugh-map}\vspace{-4mm}\\
 Canonic Form \\ 
 M = A'D
\\ \phantom\\ \\ \phantom\\ 
 \underline{KMAP for output E: } \\\vspace{-8mm}
\begin{karnaugh-map}[4][4][1][][][][]
        \minterms{2,6,8,9,10,11,14}
        \terms{}{$x$}
        \autoterms[0]
        \implicant{2}{10}
        \implicant{8}{10}
        \draw[color=black, ultra thin] (0, 4) --
    node [pos=0.7, above right, anchor=south west] {$CD$} % Y label
    node [pos=0.7, below left, anchor=north east] {$AB$} % X label
    ++(135:1);
\end{karnaugh-map}\\ 
 Canonic Form\\
E = CD' + AB' 


\\\underline{KMAP for output L: } \\ \vspace{-4mm}
\begin{karnaugh-map}[4][4][1][][][][]
        \minterms{0,4,12}
        \terms{}{$x$}
        \autoterms[0]
        \implicant{0}{4}
        \implicant{4}{12}
        \draw[color=black, ultra thin] (0, 4) --
    node [pos=0.7, above right, anchor=south west] {$CD$} % Y label
    node [pos=0.7, below left, anchor=north east] {$AB$} % X label
    ++(135:1);
\end{karnaugh-map}\\
Canonic Form\\
L = A'C'D' + BC'D' 
\subsection*{Coin detector}

The inputs for this design will be photocell representations that will be strategically placed in order to detect what type of coin is inserted into the vending machine by changing their value from (0) to (1) where when input C is covered will be when a \$1 coin is inserted, input B and C will be covered in case a \$5 coin is inserted and lastly, inputs A, B and C will be covered if a \$10 coin is inserted. \\ \\ \phantom\\ 
This design will have four outputs, the first three outputs will represent successful combinations that represent each of the expected coins, "UP" for \$1, "CP" for  \$5 and lastly  "DP" for \$10. In the other hand an output labeledd "M" will represent an unexpected combination, which would mean the case that a photocell detects a counterfeit coin that does not meet the preset conditions.\\ \\ \phantom\\
The procedure that was carried out in order to develop the code that executes the design was to implement the inputs and outputs mentioned above and to create a code in VHDL together with Karnaugh maps that served to find the most simplified version of the Boolean function that the code needs. In this way it was detected when the output gave answer 1 or 0 giving logic to the solution.
\subsubsection*{Karnaugh maps}
\underline{KMAP for output UP: }\\
\begin{karnaugh-map}[2][4][1][][][][]
        \minterms{1}
        \terms{}{$x$}
        \autoterms[0]
        \implicant{1}{1}
        \draw[color=black, ultra thin] (0, 4) --
    node [pos=0.7, above right, anchor=south west] {$C$} % Y label
    node [pos=0.7, below left, anchor=north east] {$AB$} % X label
    ++(135:1);
\end{karnaugh-map}
\vspace{-0.5cm}
\flushleft Canonic Form \\ 
UP = A'B'C\\ 

\vspace{3cm}
\underline{KMAP for output CP: }\\
\begin{karnaugh-map}[2][4][1][][][][]
        \minterms{3}
        \terms{}{$x$}
        \autoterms[0]
        \implicant{3}{3}
        \draw[color=black, ultra thin] (0, 4) --
    node [pos=0.7, above right, anchor=south west] {$C$} % Y label
    node [pos=0.7, below left, anchor=north east] {$AB$} % X label
    ++(135:1);
\end{karnaugh-map}\vspace{-6mm}\\
Canonic Form \\ 
CP = A'BC \\
\vspace{0.2cm}
\underline{KMAP for output DP: }\\
\begin{karnaugh-map}[2][4][1][][][][]
        \minterms{7}
        \terms{}{$x$}
        \autoterms[0]
        \implicant{7}{7}
        \draw[color=black, ultra thin] (0, 4) --
    node [pos=0.7, above right, anchor=south west] {$C$} % Y label
    node [pos=0.7, below left, anchor=north east] {$AB$} % X label
    ++(135:1);
\end{karnaugh-map} \vspace{-4mm} \\

Canonic Form \\ 
DP = ABC \\
\vspace{4mm}
\underline{KMAP for output M: }\\
\begin{karnaugh-map}[2][4][1][][][][]
        \minterms{2,4,5,6}
        \terms{}{$x$}
        \autoterms[0]
        \implicant{2}{6}
        \implicant{4}{5}
        \draw[color=black, ultra thin] (0, 4) --
    node [pos=0.7, above right, anchor=south west] {$C$} % Y label
    node [pos=0.7, below left, anchor=north east] {$AB$} % X label
    ++(135:1);
\end{karnaugh-map}\\ 
Canonic Form \\ 
M = BC'+AB' \\

\section*{Analysis}
\justifying
After performing the corresponding procedures to carry out the designs proposed to the practice it can be corroborated that the results obtained in the practice were as expected compared to the theory, since after the implementation of the program to the hardware that in this case is the Basys 3 board, the representation of (0) and (1) in this case were the LEDS, which were lit when expected. \\

The sample of the practice where you will see the LEDs of the Basys 3 board will be in the images folder, inside it there will be two folders each one with the name corresponding to its design.



\section*{EDA Playground links}
\begin{itemize}
    \item \textbf{Cistern} \url{www.edaplayground.com/x/GUYn}
    \item \textbf{Coin detector} \url{www.edaplayground.com/x/8GnY}
\end{itemize}
\section*{Conclusion}
This report was a success because first of all the proposed objective has been met, since the proposed designs were successfully implemented with the knowledge learned during the first period, such as logic gates, simplification through the use of Karnaugh maps and truth tables, with this we were able to program combinational logic circuits and thus develop functions based on the planted exercises that when implemented in EDAPlayground gave the expected results, This program was used to program the switches and LEDs of the Basys 3 board for its subsequent loading, where it was confirmed that what was proposed in the design and programmed through the software gave the expected result in the hardware used.

. \raggedcolumns
    
%Comando para citar  \cite{gh1562}

% Bibliografía 
%---------------------------------------------------------------------------------------------------------------------------------------------------------------
\newpage
\bibliographystyle{bst/IEEtran} %Estilo de bibliografía NO MODIFICAR PARA MANTENER FORMATO
\bibliography{bib/bibliografia} 

%Fuentes bibliográficas Se recomienda utilizar un gestor de referencias (zotero, jabref, etc..)

% ----------------------------------------------------------------------------------------------------------------------------------------------------------------
\end{multicols}
\end{document}	
